\section{Introduction and motivation}
The task chosen for this project report is to find out, whether a Convolutional Neural Network (CNN) is capable of accurately detecting instances of driver inattention. 
In order to achieve this task and answer this self-imposed question this report will cover the major steps of training a cnn on visual data of various states of driver attention.  
\autoref{sec:1} will go into detail on the chosen approach for training and optimizing the CNN's performance and will briefly present the results of the hyperparameter optimization. \autoref{sec:2} will go into detail about the data augmentation and the precautions in place to prevent overfitting and will be the focus of this report.
\autoref{sec:3} showcases the results of an alternative method and compares them to the results our method achieved.
Hyperparameter optimization is presented in greater detail in Lennart Völz's Project Report \cite{lennard} and will therefore only be briefly covered in this report. 
Driver inattention is one of the leading causes of road safety violations and can have different causes, like tiredness, calling, or even drinking while driving. For this reason, our goal is to not only detect instances of inattention 
but also detect different forms of inattention in a multi-class classification problem. We believe that driver attention is of great importance for the safety of every traffic participant and offers a great opportunity for improvement via 
modern machine learning techniques. All plots and values presented in this report were obtained by the code stored in the Github repository \cite{github}.