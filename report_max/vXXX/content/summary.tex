\newpage
\section{Summary}
This report successfully showed that a CNN is capable of successfully detecting instances of driver inattention and distinguishing between different kinds of inattention. 
To arrive at this result, data processing and image augmentation were used to combat large computational costs and imbalanced classes. Hyperparameter optimization by doing a random and iterative Bayesian search yielded a complex and well-working model architecture with finetuned parameters, producing high-precision results on the initial and augmented data. A 12-fold cross-validation on the augmented data validates the generalization of the results and outperforms the alternative model, which uses significantly higher resolution data than the final CNN but fails at the classification task in some instances. Even on the imbalanced data, where the CNN's performance is not as good as on the augmented data, the alternative method is outperformed by the CNN because of its incapability to do classification in some instances.
The reason for the classification of some images as unidentified might be due to a lack of confidence in the Roboflow model. With no known method in the imported model to set the confidence threshold, these results produced by the Roboflow model had to be worked with. It was decided, that excluding instances where classification failed from the metrics is bad practice. This leads to a significant decrease in alternative model performance. With no way of setting a confidence threshold or insight as to why the classification fails in some instances, the CNN outperforms the Roboflow model on both the original data and especially when trained on balanced classes. When excluding unidentified images from the Roboflow metrics, the model shows an immense performance increase, which could not be matched by the CNN. With more images of higher resolution, this performance might be matchable. With computational power and time constraining the amount of data usable for the training, 91\% average precision and recall are the final results of this report. This can be labeled a huge success considering the large amount of data reduction happening, showing that the used CNN is highly capable of feature extraction to achieve such high scores on the limited data it was provided.