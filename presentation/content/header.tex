\PassOptionsToPackage{unicode}{hyperref}
\documentclass[aspectratio=1610, professionalfonts, 9pt]{beamer}

\usefonttheme[onlymath]{serif}
\usetheme[showtotalframes]{tudo}
%Figure label with "Figure #.:..."
%\setbeamertemplate{caption}[numbered]

\setbeamertemplate{caption}{\insertcaption}

\usepackage{polyglossia}
\setmainlanguage{english}
\setotherlanguages{german}
\usepackage{blindtext}

\usepackage{booktabs}
\usepackage[autostyle]{csquotes}

\usepackage{appendixnumberbeamer}

% Mathematik
\usepackage{multicol}
\usepackage{graphicx}
\usepackage{amssymb}
\usepackage{amsmath}
\usepackage{mathtools}
\usepackage{fontspec}
\usepackage[version=4]{mhchem}
\usepackage{xparse}
\usepackage{braket}
% \usepackage{units}
\usepackage[locale=UK, separate-uncertainty=true, per-mode=reciprocal]{siunitx}
\sisetup{math-micro=\text{µ},text-micro=µ}
\DeclareSIUnit\px{px}
\DeclareSIUnit\parsec{pc}
\usepackage[section]{placeins}
\usepackage{pdflscape}
\usepackage{expl3}
\usepackage{bookmark}
%Komma als Dezimaltrenner in der mathe Umgebung, um in Umgebungen wie [0, 2] ein Leerzeichen nach dem Komma zu erhalten einfach eins setzen
\usepackage{icomma}
\usepackage{cancel}
\usepackage{tikz}
% \usepackage{feynman-tikz}
\usepackage{hyperref}
\usepackage{bookmark}
\usepackage{nicefrac}
\usepackage[format=plain,
            textfont={it, scriptsize}]{caption}
%\usepackage{subfigure}
\usepackage{subcaption}
\usepackage{float}
\floatplacement{table}{htbp}
\floatplacement{figure}{htbp}
\usepackage[
  backend=biber,
  autolang=hyphen,
  sorting=none,
]{biblatex}
%\usepackage{atlasbiblatex}
\addbibresource{lit.bib}  % die Bibliographie einbinden
\DefineBibliographyStrings{english}{andothers = {{et\,al\adddot}}}
\usepackage{xpatch}
\xapptocmd\citesetup{\tiny}{}{}

\usepackage{mhchem}

%\makeatletter
\defbeamertemplate*{note page}{mynotes}
{%
  {%
    \scriptsize
    \usebeamerfont{note title}\usebeamercolor[fg]{note title}%
    \ifbeamercolorempty[bg]{note title}{}{%
      \insertvrule{.45\paperheight}{note title.bg}%
      \vskip-.45\paperheight%
      \nointerlineskip%
    }%
    \vbox{
      \hfill\insertslideintonotes{0.5}\hskip-\Gm@rmargin\hskip0pt%
      \vskip-0.75\paperheight%
      \nointerlineskip
      \begin{pgfpicture}{0cm}{0cm}{0cm}{0cm}
        \begin{pgflowlevelscope}{\pgftransformrotate{90}}
          {\pgftransformshift{\pgfpoint{-4cm}{0.2cm}}%
          \pgftext[base,left]{\usebeamerfont{note date}\usebeamercolor[fg]{note date}\the\year-\ifnum\month<10\relax0\fi\the\month-\ifnum\day<10\relax0\fi\the\day}}
        \end{pgflowlevelscope}
      \end{pgfpicture}}
    \nointerlineskip
    \vbox to .75\paperheight{\vskip0.5em\vfil}%
  }%
  \ifbeamercolorempty[bg]{note page}{}{%
    \nointerlineskip%
    \insertvrule{.55\paperheight}{note page.bg}%
    \vskip-.55\paperheight%
  }%
  \vskip.25em
  \nointerlineskip
  \insertnote
}
\makeatother

\makeatletter
\defbeameroption{show only notes}[]%
{
  \beamer@notestrue
  \beamer@notesnormalsfalse
}
\makeatother


\usepackage{pgfpages}
%\setbeameroption{show notes on second screen=right}
\setbeameroption{show only notes}
\setbeamertemplate{note page}[mynotes]

%\usepackage{pgfpages}
%\setbeameroption{show notes on second screen}
%\setbeameroption{show only notes}

%define some abbreviations so you spare some time writing
\newcommand\elec{$\mathrm{e^{-}}$}
\newcommand\he{\ce{^{3}He} }
\newcommand\boronc{\ce{B_{4}C} }
\newcommand\alphap{$\alpha$-particle }
\newcommand\alphaps{$\alpha$-particles }
\newcommand\boront{\ce{^{10}B} }
\newcommand\borone{\ce{^{11}B} }
\makeatletter
\let \@@magyar@captionfix\relax
\makeatother



%%%%%%%%%%%%%%%%%%%%%%%%%%%%%%%%%%%%%%%%%%%%%%%%%%%%%%%%%%%%%%%%%%%%%%%%%%%%%%%%
%%%%%-------------Hier Titel/Autor/Grafik/Lehrstuhl eintragen--------------%%%%%
%%%%%%%%%%%%%%%%%%%%%%%%%%%%%%%%%%%%%%%%%%%%%%%%%%%%%%%%%%%%%%%%%%%%%%%%%%%%%%%%

%Titel:
\title{Driver inattention detection}

%Autor
\author[M.~Möller, L.~Völz]{Max Möller, Lennart Völz}
%Lehrstuhl/Fakultät
\institute[]{TU Dortmund University, Machine Learning for Physicists}
